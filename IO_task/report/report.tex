\documentclass[a4peper, 12pt, titlepage, finall]{extreport}

%различные пакеты

\usepackage[T1, T2A]{fontenc}
\usepackage[russian]{babel}
\usepackage[justification=centering]{caption}
\usepackage[backend=bibtex]{biblatex}
\usepackage{csquotes}
\usepackage{tikz}
\usepackage{geometry}
\usepackage{indentfirst}
\usepackage{fontspec}
\usepackage{graphicx}
\usepackage{array}
\usepackage{pgfplots}
\graphicspath{{./images/}}

\usetikzlibrary{positioning, arrows}

\geometry{a4paper, left = 15mm, top = 10mm, bottom = 15mm, right = 15mm}
\setmainfont{Spectral Light}%{Times New Roman}
\setcounter{secnumdepth}{0}
%\setcounter{tocdepth}{3}
\nocite{*}
\begin{document}
\begin{center}
    {\large \bf Экспериментальное исследование разработанного алгоритма имитации отжига в рамках второго задания по курсу 
    «Прaктикум на ЭВМ.»}

\end{center}
        \begin{flushright}
            {Никифоров Никита Игоревич, 421 группа}\\
        \end{flushright}
    \section{Введение}
        Необходимо было разработать алгоритм имитации отжига в применении к задаче построения распределённого рассписания. 
        В моём варианте необходимо было реализовать критерий К3, а именно разбалансированность расписания 
        (т.е. значение разности Tmax$-$Tmin, где Tmax $-$ наибольшая, по всем процессорам, длительность расписания на процессоре; Tmin $-$ аналогично, наименьшая длительность).
        Также необходимо было реализовать многопоточный алготм имитации отжига с синхронизацией через разделяемую память.
    
    \section{Методика экспериментального исследования}
        Все тестовые данные используемые в данном экспериментальном исследовании доступны в папке tests проекта. 
        Так как алгоритм является недетерменированным бралась средняя оценка по критерию среди 10 прогонов. 
        Время выполнения фиксировалось утилитой c с помощью библиотеки $std::chrono$.
        Исследование проводилось без запущенных посторонних программ.
    
    \section{Тестовый стенд}
        \begin{itemize}
            \item Процессор $-$ AMD Ryzen 2700X 8/16 (ядер/потоков) 4.2 Ghz на одно ядро, 4.0 Ghz на все ядра.
            \item Память $-$ 32Gb DDDR4 3200Mhz (Пропускная способность 44GB/s).
        \end{itemize}

    \section{Экспериментальное исследование}
        \subsection{Поиск опорной точки}
            Необходимо найти опорную точку экспериментального исследования, а именно вычислить количество задач и процессоров, такое, что время выполения на одном потоке будет
            больше 1 минуты при использовании какого либо закона понижения температуры.
            
            \begin{figure}[!htbp]
                \centering
            \begin{tikzpicture}
            \begin{axis}[
                width=0.8\textwidth,
                height=0.6\textwidth,
                ylabel={Время, секунды},
                xlabel={Количество задач, шт},
                y label style={at={(-0.05, 0.5)}},
                ymin=0, ymax=100,
                xmin=0, xmax=40000,
                xtick={0, 100, 1000, 5000, 10000, 15000, 20000, 25000, 30000, 35000, 40000},
                ytick={0, 5, 10, 15, 20, 25, 30, 35, 40, 45, 50, 55, 60, 65, 70, 75, 80, 85, 90},
                legend pos=south east,
                ymajorgrids=true,
                grid style=dashed,
                cycle list name=color list] 
                \addplot+[mark=square] coordinates{(100, 0.124)(1000, 0.845)(5000, 1.240)(10000, 4.153)(15000, 9.32)(20000, 16.068)(25000, 24.837)(30000, 36.072)(35000, 49.366)(40000, 71.032)};
                \addplot+[mark=square] coordinates{(100, 0.005)(1000, 0.095)(5000, 1.418)(10000, 4.774)(15000, 11.045)(20000, 20.359)(25000, 32.257)(30000, 45.22)(35000, 55.929)(40000, 73.896)};
                \addplot+[mark=square] coordinates{(100, 0.004)(1000, 0.070)(5000, 1.208)(10000, 4.213)(15000, 9.270)(20000, 15.746)(25000, 27.844)(30000, 35.656)(35000, 52.656)(40000, 68.522)};
                \legend{Коши, Больцмана, смешанный закон}
     
            \end{axis}
            \end{tikzpicture}
                \captionsetup{justification=centering}
                \caption{Зависимость времени исполнения от количества задач}
                \label{graph:root_point}
            \end{figure}
            
            Таким образом на графике~\ref{graph:root_point} мы видим, что алгоритм ведёт себя одинаково в рамках погрешности при изменении закона понижения температуры. 
            Также мы видим, что для всех законов время выполнения заходит за минуту только на 40000 задач. Соответственно в дальнейшем я буду использовать именно это количество.
            %\begin{figure}[!htbp]
            %    \centering
            %\begin{tikzpicture}
            %\begin{axis}[
            %    width=0.4\textwidth,
            %    height=0.4\textwidth,
            %    ylabel={Лучшая оценка, попугаи},
            %    xlabel={Количество задач, шт},
            %    y label style={at={(-0.05, 0.5)}},
            %    ymin=0, ymax=100,
            %    xmin=0, xmax=64000,
            %    xtick={0, 100, 1000, 5000, 10000, 15000, 20000, 25000, 30000, 35000, 40000},
            %    ytick={0, 5, 10, 15, 20, 25, 30, 35, 40, 45, 50, 55, 60, 65, 70, 75, 80, 85, 90},
            %    legend pos=south east,
            %    ymajorgrids=true,
            %    grid style=dashed,
            %    cycle list name=color list] 
            %    \addplot+[mark=square] coordinates{(100, )(1000, )(5000, )(10000, )(15000, )(20000, )(25000, )(30000, )(35000, )(40000, )};
            %\legend{}
            %
            %\end{axis}
            %\end{tikzpicture}
            %    \captionsetup{justification=centering}
            %    \caption{Dependence of memory from number of rules.}
            %    \label{graph:btmem}
            %\end{figure}
        \subsection{Зависимость изменений в поведении алгоритма от количества потоков}
            Во второй части экспериментального исследования необходимо было исследовать изменения поведения алгоритма при изменении количество потоков.
            В данной части будет использован тестовый файл $test/procs/40000\_10.xml$, и закон понижения температуры Коши.
           \begin{figure}[!htbp]
               \centering
           \begin{tikzpicture}
           \begin{axis}[
               width=0.8\textwidth,
               height=0.6\textwidth,
               ylabel={Лучшая оценка, попугаи},
               xlabel={Количество процессоров, шт},
               y label style={at={(-0.05, 0.5)}},
               ymin=8, ymax=13,
               xmin=0, xmax=18,
               ytick={8, 8.5, 9, 9.5, 10, 10.5, 11, 11.5, 12, 12.5},
               xtick={0, 2, 4, 6, 8, 10, 12, 14, 16, 18},
               legend pos=south east,
               ymajorgrids=true,
               grid style=dashed,
               cycle list name=color list] 
           \addplot+[mark=square] coordinates{(1, 12.285)(2, 10.781)(4, 10.154)(6, 11.116)(8, 10.226)(10, 10.098)(12, 9.746)(14, 10.021)(16, 9.900)};
           \legend{40000 задач; 10 процессеров}
           
           \end{axis}
           \end{tikzpicture}
               \captionsetup{justification=centering}
               \caption{Зависимость оценки лучшего решения от количества процессов.}
               \label{graph:all_popugs}
           \end{figure}
           \begin{figure}[!htbp]
               \centering
           \begin{tikzpicture}
           \begin{axis}[
               width=0.8\textwidth,
               height=0.6\textwidth,
               ylabel={Время выполнения, секунды},
               xlabel={Количество процессоров, шт},
               y label style={at={(-0.05, 0.5)}},
               ymin=60, ymax=250,
               xmin=0, xmax=18,
               ytick={60, 80, 100, 120, 140, 160, 180, 200, 220, 240},
               xtick={0, 2, 4, 6, 8, 10, 12, 14, 16, 18},
               legend pos=south east,
               ymajorgrids=true,
               grid style=dashed,
               cycle list name=color list] 
           \addplot+[mark=square] coordinates{(1, 66.902)(2, 68.688)(4, 68.013)(6, 75.737)(8, 86.970)(10, 121.181)(12, 152.793)(14, 198.684)(16, 243.330)}; 
           \legend{40000 задач; 10 процессеров}
           
           \end{axis}
           \end{tikzpicture}
               \captionsetup{justification=centering}
               \caption{Зависимость времени выполнения от количества процессов}
               \label{graph:all_time}
           \end{figure}
            Как можно видеть по графикам ~\ref{graph:all_popugs} и ~\ref{graph:all_time}, при увеличении количества процессов до 4-х время работы алгоритма не увеличивается,
            А точность алгоритма улучшается в переходе от одного процесса на два на $20\%$. Дальнейшее улучшение производительности не было значительным, а прирост времени исполнения сильным.

    \section{Послесловие}
        В папке $tests/root\_point$ находится файл $50000\_100.xml$, если запустить программу вместе с сидом равным $1604022559$, то при использовании оптимизированного начального решения,
        алгоритм не сходится. А именно после завершения \"Параллельного\" $Nproc = 1$ алгоритма имитации отжига, оценка лучшего найденного решения равна оценке исходного $3196.39$.
        При этом алгоритм работает меньше секунды.
        Если же задать \"прямое\" начальное решение, а именно, когда все задачи находятся на одном процессоре алгоритм имитации отжига работает более 2-х минут, и находит оптимальное решение
        оценка которого равна $51.69$.

        В данном случае мы наблюдаем стандартный вариант нахождения локального минимума.


\end{document}
